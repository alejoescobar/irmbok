\chapter{About the Society of Information Risk
Analysts}\label{about-the-society-of-information-risk-analysts}

\section{Who We Are}\label{who-we-are}

The Society of Information Risk Analysts (SIRA) \index{SIRA}is a rapidly
growing, global non-profit association established to advance the full
maturation and mainstream acceptance of Information Risk Management
(IRM) as a discipline and profession. SIRA is a professional association
which develops, maintains, and disseminates pragmatic, realistic,
implementable, evidence-based IRM practices and methodologies.

\section{Contact Us}\label{contact-us}

For more information, please contact the Society of Information Risk
Analysts at:

\begin{quote}
Society of Information Risk Analysts 1023 Delaware Avenue\\ Mendota
Heights, MN 55118 USA\\ Website: \url{http://www.societyinforisk.org}\\
Contact Form: \url{https://www.societyinforisk.org/contact}
\end{quote}

\chapter{Introduction}\label{introduction}

\section{What Is the Information Risk Management Body of
Knowledge?}\label{what-is-the-information-risk-management-body-of-knowledge}

The Information Risk Management Body of Knowledge (iRMBOKTM) is a
standard for the practice of information risk management.
\index{Risk Management} The iRMBOKTM Guide describes information risk
management tasks and the knowledge required to be effective.

The primary goal of the iRMBOKTM Guide is to define the emerging
profession of information risk management. It serves as a baseline that
practitioners can agree upon in order to discuss the work they do and to
ensure that they have the skills they need to effectively perform the
role, and defines the skills and knowledge that people who work with and
employee information risk managers should expect a skilled practitioner
to demonstrate. It is a framework that describes the information risk
management tasks that must be performed in order to understand how an
organization may manage its information risks.

This chapter provides an introduction to key concepts in the field of
information risk management and describes the structure of the remainder
of the iRMBOK Guide.

\section{What is Information Risk
Management?}\label{what-is-information-risk-management}

Information Risk Management (IRM) is the continuous process of
information risk analysis (see Chapter 2: Risk Assessment), information
risk treatment (see Chapter 3: Risk Treatment), and information risk
communication (see Chapter 4: Communication and Consultation).

\section{Is Information Risk Management a Failed
Concept?}\label{is-information-risk-management-a-failed-concept}

While IRM has had its share of critics, probably none has been as vocal
as Donn Parker. In a 2006 article in the journal of the Information
Systems Security Association, Parker claimed that risk-based security is
a ``failed concept.''1 Parker thinks that there is too much uncertainty
and complexity in the data regarding rare incidents to effectively apply
the principles of decision theory to information security. Furthermore,
even if these obstacles could be overcome, Parker believes ``it is too
easy for management to accept security risk rather than reducing it by
increasing security that is inconvenient and interferes with
business.''2 Given Parker's well-earned influence in the information
security community and the importance of the topic, it's worth
considering his objections in detail.

In his 2006 article, Parker proposes that risk-based security ``\ldots{}
must be replaced with practical, doable security management with the new
objectives of due diligence, compliance consistency, and enablement''
(hereafter, ``diligence-based security'').3 Due diligence is necessary
to avoid negligence; compliance to avoid penalties; and enablement to be
competitive. According to Parker, ``Reduction of security risk then
becomes serendipitous.'' In his more recent writings, Parker has avoided
any implications that his diligence method attempts to meet or is
related to the legal concept of due diligence, by referring to diligence
only.

Here we need to be careful to distinguish three options regarding the
foundation for information security management:

\begin{itemize}
\itemsep1pt\parskip0pt\parsep0pt
\item
  Information security management based upon diligence-based security
  only.
\item
  Information security management based upon risk-based security only.
\item
  Information security management based upon both risk- and
  diligence-based security.
\end{itemize}

Parker endorses (I), whereas I shall argue for (III). What is striking
about Parker's article is that he writes as if the only options were (I)
and (II), but that is a false dichotomy. It seems to me that diligence-
and risk-based security are complementary, in four ways.

First, compliance often requires risk-based security. As Parker himself
now admits, some laws require information security risk analysis (ISRA).
In the U.S., such laws include the Federal Trade Commission Act (``FTC
Act''),5 Gramm-Leach-Bliley Act (GLBA),6 the Federal Information
Security Management Act of 2002 (FISMA),7 etc. Outside of the U.S., such
laws include the EU Data Protection Directive8 and Japan's Personal
Information Protection Act.9 Furthermore, contractual obligations can
require an organization to perform a formal risk assessment. For
example, many organizations are contractually obligated to comply with
the Payment Card Industry (PCI) Security Standards Council's Data
Security Standard (DSS), which in turn requires an annual risk analysis
(RA).10 Along the same lines, any organizations subject to any of the
U.S. state laws mandating PCI DSS compliance11 arguably have a duty to
perform an RA.

Second, business enablement should properly be taken into account in an
effective RA, as failing to enable the business amounts to a failure to
achieve business objectives, which is broadly equivalent to risk.

Third, new and emerging security threats are especially problematic if
one eschews a risk-based approach to security. When a new threat is
discovered, laws are unlikely to mandate specific security controls to
deal with that threat. Moreover, due diligence is unlikely to be helpful
either, since there probably will not be any de facto industry standard
for mitigating that threat. For example, laptop encryption is surely
part of the standard of care today, but it was not five years ago.
Additionally, compliance is often the lowest common denominator that
protects entities outside of an organization (e.g., consumers,
government, payment card brands) more than the organization itself.
Compliance is rarely an efficient way of allocating resources, since
compliance requirements are rarely (if ever) designed with a specific
organization's nuances in mind.

Fourth, there are often multiple options that can be used to avoid
negligence, achieve legal compliance, and enable the business.
Organizations have limited resources to invest in information security.
Sometimes the resources required just to achieve compliance, due
diligence, and enablement exceed the resources that are available.
Lawmakers and other organizations need a decision making method for
selecting one of those options. The methods of decision theory,
including risk analysis, are empirically well supported. The
diligence-based method is not.

Furthermore, while Parker has modified his position by removing an
appeal to the legal concept of due diligence, that does not deny the
fact due diligence to avoid negligence itself requires a risk- based
approach. Let us define due diligence as the prudent person's
fulfillment of the duty to use reasonable care; negligence is the
failure to do so. In this context, ``reasonable care'' means ``such care
as what a reasonable prudent and careful person would use under similar
circumstances.''

In United States v. Carroll Towing Co, Judge Learned Hand determined
that a party has a duty to take adequate measures to prevent harm if the
cost (B) of taking adequate measures to prevent harm is less than the
monetary loss (L) multiplied by the probability (P) of its occurring,
expressed by the equation ``B \textless{} PL.'' Thus, the due diligence
needed to determine a party's duty of care requires a cost- benefit
analysis that weighs the risk (P × L) against the cost (B) to mitigate
that risk.

Moreover, as explained by U.S. Supreme Court Justice Holmes in Texas \&
P.R. v Behymer, ``{[}w{]}hat usually is done may be evidence of what
ought to be done, but what ought to be done is fixed by a standard of
reasonable prudence, whether it is usually complied with or not.''17
There is strong evidence that risk-based security is the standard.
First, several high-profile U.S. organizations have publicly endorsed
security risk management, including the Government Accountability
Office,18 the Federal Trade Commission,19 the U.S. Marine Corps,20 the
U.S. Air Force,21 and Microsoft.22 Second, a number of professional
societies and standards bodies advocate security risk management,
including the Information Systems Security Association (ISSA),23 the
Information Systems Audit and Control Association (ISACA),24 the
American Society for Industrial Security (ASIS),25 the Institute of
Internal Auditors (IIA),26 Standards Australia and Standards New
Zealand,27 the British Standards Institute,28 the U.S. National
Institute of Standards and Technology,29 and, most notably, the
International Organization for Standardization.

As I read him, Parker's critique of risk-based security consists of
eight supporting arguments. Those arguments may be divided into three
categories: theoretical, empirical, and practical. Let us examine each
of his supporting arguments in turn.

\subsection{Theoretical Arguments against Risk-Based Information
Security}\label{theoretical-arguments-against-risk-based-information-security}

Let's begin by considering Parker's arguments against the possibility of
actually doing risk-based security in the real world.

\subsubsection{First Supporting Argument: Uncertainties Involved in
ISRA}\label{first-supporting-argument-uncertainties-involved-in-isra}

Here is Parker:

\begin{quote}
The frequencies and impacts of future incidents are under the control of
unknown and often irrational enemies with unknown skills, knowledge,
resources, authority, motives, and objectives from unknown locations at
unknown future times attacking known but untreated
\textgreater{}vulnerabilities and vulnerabilities that are known to the
attackers but unknown to the defenders (a constant problem in our
\textgreater{}technologically complex environments).
\end{quote}

This objection to risk-based security is multiply flawed.

First, many of the variables listed by Parker are simply not relevant to
assessing the probability of an attack. One does not need to know the
identity of an attacker, much less his ``skills, knowledge, resources,
authority, motives and objectives'' (SKRAMO), in order to estimate the
probability of an attack. There is no doubt that we often lack knowledge
about the SKRAMO of our attackers, but that doesn't mean we cannot
calculate the probability of an attack.

Suppose we have historical data about the frequency of occurrence a
particular type of security threat spanning multiple years, across
multiple organizations of varying size, geographical location, and so
forth. For example, let the threat be workplace violence, a threat that
involves ``irrational, unknown humans.'' Based on the historical data
just mentioned, one can infer a statistical generalization about the
frequency of the workplace violence threat overall. One can also make
more specific generalizations about various subsets of the overall
workplace violence threat. For example, one can make statistical
generalizations about the rate of incidents of workplace violence in
organizations that recently went through a round of layoffs, in
individuals who were subject to one or more negative personnel actions,
and so forth.

My claim is that, as a security professional, I can use that statistical
data in a quantitative RA of the workplace violence threat for my
company. Yes, there are differences between my company and the other
organizations for which we have statistical data about incidents of
workplace violence. But the mere existence of such differences doesn't
automatically invalidate inductively correct or statistically valid
inferences about the level of risk for my company, given what we know
about the rate of occurrence of workplace violence in other companies.
In order for such inferences to be inductive incorrect or statistically
invalid, one would have to show that the differences between my company
and other companies are probabilistically relevant.

Suppose Microsoft announces tomorrow the existence of a previously
unknown security vulnerability in one of their software products. There
won't be any historical data tomorrow regarding the rate of occurrence
of attempted exploits of that vulnerability. There won't be any
historical data tomorrow regarding our enemies and whether they are
planning on exploiting that vulnerability. Despite that lack of data,
however, we can still make accurate calculations of the level of risk,
based upon what we know about past security vulnerabilities and past
enemy attacks.

Second, even in those situations where historical data about the
statistical frequency of a specific threat is unavailable, historical
data is often available for some larger class of events for which the
specific threat is a member. Every time a new vulnerability is announced
in a given piece of software, by definition there will be no historical
data about that vulnerability. Yet we have a large amount of statistical
data about information security vulnerabilities in general. We also have
a wealth of historical information about vulnerabilities affecting
specific types of software. For example, the next time a new
vulnerability in the Apache web server software is announced, we won't
have any statistical data regarding the frequency of exploits of that
specific vulnerability, but we will have data about the frequency of
past Apache vulnerabilities for which exploit code is publicly
available. That latter data is relevant to determining the probability
that exploit code will be made publicly available for the new
vulnerability.

A similar point applies to the worry about threats stemming from the
acts of individuals (as opposed to ``acts of nature''). While we may not
have any historical data about the probability of a specific enemy
committing attack Y, we do have statistical data about attacks in
general, specific types of attacks, and attacks against specific
organizations. To be sure, the more specific the reference class, the
more confidence we will have in our probability values. Just because our
reference class is not identical to the event in question, however, it
does not follow that we cannot have a reasonable or even high degree of
confidence in our probability values. The fact that we cannot know
something with certainty (i.e., probability = 100\%) does not prevent us
from knowing it with a high degree of probability (i.e., probability
\textgreater{} 50\%).

Parker also introduces the following related objection:

\begin{quote}
In addition, when enemies fail in attacking one possible vulnerability,
they often attempt attacks on other vulnerabilities to
\textgreater{}accomplish their goals. Therefore, risks may be related in
unknown complex ways so that reducing one risk may increase or decrease
\textgreater{}other risks. This alone precludes the effective use of
risk assessment methods.32
\end{quote}

Parker is certainly correct that if an initial attempt to exploit a
vulnerability fails, many attackers will try other attacks in order to
accomplish their goals. It does not follow, however, that this fact
``alone precludes the effective use of risk assessment methods''
(emphasis mine). This is an incredibly strong claim that requires a
supporting argument from Parker, but such an argument is not provided in
his article.

In sum, far from disqualifying the use of RA, our uncertainty about
attackers provides a strong reason for using probabilistic, risk-based
methods. As Doug Hubbard writes, ``We use probabilistic methods because
we lack perfect data, not in spite of lacking it. If we had perfect
data, probabilities would not be required.''33 Furthermore, ``It is a
fallacy that when a variable is highly uncertain, we need a lot of data
to reduce the uncertainty. The fact is that when there is a lot of
uncertainty, less data is needed to yield a large reduction in
uncertainty.''

\subsubsection{Second Supporting Argument: Complex, Unknowable
Relationships between Risks and Security
Efforts}\label{second-supporting-argument-complex-unknowable-relationships-between-risks-and-security-efforts}

Parker's next argument claims that that the relationships between risks
and security efforts are complex and often not completely knowable:
